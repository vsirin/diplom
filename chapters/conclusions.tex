\newpage
\begin{center}
\noindent\textbf{ЗАКЛЮЧЕНИЕ}\label{chapters:conclusions}
\vspace{1.5mm}
\end{center}

В данной работе рассматривалась задача о построении раскрасок двумерных сфер с запрещенными единичными расстояниями. 
В ходе работы были полностью решены все поставленные задачи и получены следующие эмпирические и теоретические результаты:

\begin{enumerate}

\item Установлено, что семейство раскрасок в $8$ и $9$ цветов можно получить на основе сферических диаграмм Вороного, соответствующих локальным минимумам задачи Томсона.  

\item Дан обзор алгоритмов и методов, применяемых при построении современных \textit{SAT}-решателей.

\item Разработан программнай код, конструирующий корректные раскраски двумерной сферы на основе решения задачи Томсона.

\item Получены раскраски и оценки хроматических чисел сфер для диапазонов радиусов.

\item Получены оценки хроматических чисел двойственных графов для регулярных решений задачи Томсона.

\item Разработана программа для визуализации сферических диаграмм Вороного и их раскрасок.

\end{enumerate}

Программный код и все материалы, необходимые для воспроизведения полученных результатов, размещены в открытом доступе по адресу
\url{https://github.com/xm-repo/sphere}.
Несмотря на то, что данная работа является еще одним небольшим шагом к решению задачи о раскраске сфер, полученные результаты нельзя назвать исчерпывающими. Среди открытых вопросов и направлений для дальнейших исследований можно отметить следующие:

\begin{enumerate}

\item Доказательство оценки для хроматического числа квадрата двойственного графа регулярных триангуляций сферы.
\item Доказательство общей оценки хроматического числа сферы для достаточно больших значений радиуса.
\item Постановка и решение аналогичных задач для трехмерных сфер.
\item Изучение случая сферы с радиусом $\frac{1}{2}+\varepsilon$.

\end{enumerate}
