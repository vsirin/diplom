
\definecolor{codegreen}{rgb}{0,0.6,0}
\definecolor{codegray}{rgb}{0.5,0.5,0.5}
\definecolor{codepurple}{rgb}{0.58,0,0.82}
\definecolor{backcolour}{rgb}{0.95,0.95,0.92}

\lstdefinestyle{mystyle}{    
    keywordstyle=\color{magenta},
    commentstyle=\color{codegreen},
    lineskip=-1ex,
    numberstyle=\tiny\color{codegray},
    stringstyle=\color{codepurple},
    basicstyle=\ttfamily\footnotesize,
    breakatwhitespace=false,         
    breaklines=true,                 
    captionpos=b,                    
    keepspaces=true,                 
    numbers=left,                    
    numbersep=5pt,                  
    showspaces=false,                
    showstringspaces=false,
    showtabs=false,                  
    tabsize=2,
    xleftmargin=0cm
}

\lstset{style=mystyle}


\newpage
\begin{center}
\noindent\textbf{ПРИЛОЖЕНИЕ 1. ДИАПАЗОНЫ ЗНАЧЕНИЙ РАДИУСА ДЛЯ РЕШЕНИЙ ЗАДАЧИ ТОМСОНА}\label{attachments:1}
\vspace{1.5mm}
\end{center}

\begin{longtable}{ccll} 
\hline
n & k & $r_{min}$ & $r_{max}$ \\ 
\hline
\endhead
\input{chapters/attachments/colorings2.txt}
\hline
\end{longtable}

\newpage
\begin{center}
\noindent\textbf{ПРИЛОЖЕНИЕ 2. КОДИРОВАНИЕ ЗАДАЧИ РАСКРАСКИ ГРАФА}\label{attachments:2}
\vspace{1.5mm}
\end{center}

\lstinputlisting[language=Python]{chapters/attachments/g2cnf.py}

\newpage
\begin{center}
\noindent\textbf{ПРИЛОЖЕНИЕ 3. ВЫЧИСЛЕНИЕ ГРАНИЦ ДИАПАЗОНОВ ЗНАЧЕНИЙ РАДИУСА}\label{attachments:3}
\vspace{1.5mm}
\end{center}

\lstinputlisting[language=Python]{chapters/attachments/build_gs.py}

\newpage
\begin{center}
\noindent\textbf{ПРИЛОЖЕНИЕ 4. ПОСТРОЕНИЕ ДИАГРАММЫ ВОРОНОГО С ИКОСАЭДРАЛЬНОЙ СИММЕТРИЕЙ}\label{attachments:4}
\vspace{1.5mm}
\end{center}

\lstinputlisting[language=Python]{chapters/attachments/sphere_triang.py}

\newpage
\begin{center}
\noindent\textbf{ПРИЛОЖЕНИЕ 5. ВИЗУАЛИЗАЦИЯ РАСКРАСОК}\label{attachments:5}
\vspace{1.5mm}
\end{center}

\lstinputlisting[language=C++]{chapters/attachments/Utils.hpp}

\lstinputlisting[language=C++]{chapters/attachments/sphere.cpp}
