
\newpage
\begin{center}
\noindent\textbf{СПИСОК ИСПОЛЬЗОВАННЫХ ИСТОЧНИКОВ}\label{chapters:biblio}
\vspace{1.5mm}
\end{center}

\begin{enumerate}[leftmargin=0.5cm,topsep=0pt,itemsep=-1ex,partopsep=1ex,parsep=1ex,ref=\arabic{*},label=\arabic{*}.]

\item\label{bib:Cook}
Cook S. A. The complexity of theorem-proving procedures //Proceedings of the third annual ACM symposium on Theory of computing. – 1971. – С. 151-158.

\item\label{bib:Karp}
Karp R. M. Reducibility among combinatorial problems //Complexity of computer computations. – Springer, Boston, MA, 1972. – С. 85-103.

\item\label{bib:Rolf}
Rolf D. Improved bound for the PPSZ/Schöning-algorithm for 3-SAT //Jour\-nal on Satisfiability, Boolean Modeling and Computation. – 2006. – Т. 1. – №. 2. – С. 111-122.

\item\label{bib:Robinson1962}
Robinson J. A. Martin Davis, George Logemann, and Donald Loveland. A machine program for theorem-proving. Communications of the ACM, vol. 5 (1962), pp. 394–397 //The Journal of Symbolic Logic. – 1967. – Т. 32. – №. 1. – С. 118-118.

\item\label{bib:MarquesSilva}
Marques-Silva J., Malik S. Propositional SAT solving //Handbook of Model Checking. – Springer, Cham, 2018. – С. 247-275.

\item\label{bib:Lardeux}
Lardeux F., Saubion F., Hao J. K. GASAT: a genetic local search algorithm for the satisfiability problem //Evolutionary Computation. – 2006. – Т. 14. – №. 2. – С. 223-253.

\item\label{bib:Selsam}
Selsam D. et al. Learning a SAT solver from single-bit supervision //arXiv preprint arXiv:1802.03685. – 2018.

\item\label{bib:HeuleBiere}
Heule M., Biere A. Proofs for satisfiability problems //All about Proofs, Proofs for all. – 2015. – Т. 55. – №. 1. – С. 1-22.

\item\label{bib:HeuleKullmann}
Heule M. J. H., Kullmann O., Marek V. W. Solving and verifying the boolean pythagorean triples problem via cube-and-conquer //International Conference on Theory and Applications of Satisfiability Testing. – Springer, Cham, 2016. – С. 228-245.

\item\label{bib:ZaikinKochemazovSemenov}
Zaikin O., Kochemazov S., Semenov A. SAT-based search for systems of diagonal latin squares in volunteer computing project sat@ home //2016 39th International Convention on Information and Communication Technology, Electronics and Microelectronics (MIPRO). – IEEE, 2016. – С. 277-281.

\item\label{bib:Järvisalo}
Järvisalo M. et al. The international SAT solver competitions //Ai Magazine. – 2012. – Т. 33. – №. 1. – С. 89-92.
	
\item\label{bib:BiereCadical}
Biere A. Cadical, Lingeling, Plingeling, Treengeling and YalSAT entering the sat competition 2018 //Proc. of SAT Competition. – 2018. – С. 13-14.

\item\label{bib:HoosStützle}
Hoos H. H., Stützle T. SATLIB: An online resource for research on SAT //Sat. – 2000. – Т. 2000. – С. 283-292.

\item\label{bib:HoosWalkSAT}
Hoos H. H., Stützle T. Local search algorithms for SAT: An empirical evalua\-ti\-on //Journal of Automated Reasoning. – 2000. – Т. 24. – №. 4. – С. 421-481.

\item\label{bib:Sorensson}
Sorensson N., Een N. Minisat v1. 13-a sat solver with conflict-clause mini\-mi\-za\-ti\-on //SAT. – 2005. – Т. 2005. – №. 53. – С. 1-2.

\item\label{bib:BiereHeule}
Biere A., Heule M., van Maaren H. (ed.). Handbook of satisfiability. – IOS press, 2009. – Т. 185.

\item\label{bib:Puri}
Puri R., Gu J. A BDD SAT solver for satisfiability testing: An industrial case study //Annals of Mathematics and Artificial Intelligence. – 1996. – Т. 17. – №. 2. – С. 315-337.

\item\label{bib:MarquesSilva1999}
Marques-Silva J. The impact of branching heuristics in propositional satisfiabi\-li\-ty algorithms //Portuguese Conference on Artificial Intelligence. – Springer, Berlin, Heidelberg, 1999. – С. 62-74.

\item\label{bib:EénBiere}
Eén N., Biere A. Effective preprocessing in SAT through variable and clause elimination //International conference on theory and applications of satisfiabi\-li\-ty testing. – Springer, Berlin, Heidelberg, 2005. – С. 61-75.

\item\label{bib:Moskewicz}
Moskewicz M. W. et al. Chaff: Engineering an efficient SAT solver //Procee\-dings of the 38th annual Design Automation Conference. – 2001. – С. 530-535.

\item\label{bib:Kullmann2009}
Kullmann O. Theory and applications of satisfiability testing //SAT. – 2009. – С. 147.

\item\label{bib:Katebi}
Katebi H., Sakallah K. A., Marques-Silva J. P. Empirical study of the anatomy of modern sat solvers //International Conference on Theory and Applications of Satisfiability Testing. – Springer, Berlin, Heidelberg, 2011. – С. 343-356.

\item\label{bib:Nadel2002}
Nadel A. Backtrack search algorithms for propositional logic satisfiability: Review and innovations. – Hebrew University of Jerusalem, 2002.

\item\label{bib:Newsham}
Newsham Z. et al. SATGraf: Visualizing the evolution of SAT formula struc\-tu\-re in solvers //International Conference on Theory and Applications of Satisfiabi\-li\-ty Testing. – Springer, Cham, 2015. – С. 62-70.

\item\label{bib:SATzilla}
Xu L. et al. SATzilla: portfolio-based algorithm selection for SAT //Journal of artificial intelligence research. – 2008. – Т. 32. – С. 565-606.

\item\label{bib:Heule2018}
Heule M. J. H. Computing small unit-distance graphs with chromatic number 5 //arXiv preprint arXiv:1805.12181. – 2018.

\item\label{bib:Soifer}
Soifer A. The mathematical coloring book: Mathematics of coloring and the colorful life of its creators. – Springer Science \& Business Media, 2008.

\item\label{bib:Kronk}
Kronk H. V., Mitchem J. A seven-color theorem on the sphere //Discrete Mathematics. – 1973. – Т. 5. – №. 3. – С. 253-260.

\item\label{bib:BruijnErdos}
Bruijn N. G., Erdos P. A colour problem for infinite graphs and a problem in the theory of relations //Indigationes Mathematicae. – 1951. – Т. 13. – С. 371-373.

\item\label{bib:Simmons}
Simmons G. J. The chromatic number of the sphere //Journal of the Austra\-li\-an Mathematical Society. – 1976. – Т. 21. – №. 4. – С. 473-480.

\item\label{bib:RaiSphere}
Raigorodskii A. M. On the chromatic numbers of spheres in $\mathbb{R}^n$ //Combinato\-ri\-ca. – 2012. – Т. 32. – №. 1. – С. 111-123.

\item\label{bib:Nech}
Oren Nechushtan. On the space chromatic number. Discrete mathematics, 256(1):499–507, 2002.

\item\label{bib:Coul}
David Coulson. A 15-colouring of 3-space omitting distance one. Discrete mathematics, 256(1):83–90, 2002.

\item\label{bib:Rai1}
Андрей М. Райгородский. О хроматическом числе пространства. Успехи математических наук, 55(2):147–148, 2000.

\item\label{bib:Larm}
David G. Larman and Ambrose C. Rogers. The realization of distances within sets in Euclidean space. Mathematika, 19(01):1–24, 1972. 

\item\label{bib:Rogers}
Rogers C. A. Covering a sphere with spheres //Mathematika. – 1963. – Т. 10. – №. 2. – С. 157-164.

\item\label{bib:Pros}
Prosanov R. Chromatic numbers of spheres //Discrete Mathematics. – 2018. – Т. 341. – №. 11. – С. 3123-3133.

\item\label{bib:Kostina}
Костина О. А., Райгородский А. М. О новых нижних оценках хроматического числа сферы //Труды Московского физико-технического института. – 2015. – Т. 7. – №. 2 (26).

\item\label{bib:ErdosGraham}
Erdos P., Graham R. L. Problem proposed at the 6th Hungarian combinato\-ri\-al conference //Eger. July. – 1981.

\item\label{bib:Larrea}
Larrea V. G. V. Construction of Delaunay Triangulations on the Sphere: A Parallel Approach. – 2011.

\item\label{bib:Thomson}
Thomson J. J. XXIV. On the structure of the atom: an investigation of the stability and periods of oscillation of a number of corpuscles arranged at equal intervals around the circumference of a circle; with application of the results to the theory of atomic structure //The London, Edinburgh, and Dublin Philosophical Magazine and Journal of Science. – 1904. – Т. 7. – №. 39. – С. 237-265.

\item\label{bib:Katanforoush}
Katanforoush A., Shahshahani M. Distributing points on the sphere, I //Ex\-pe\-ri\-men\-tal Mathematics. – 2003. – Т. 12. – №. 2. – С. 199-209.

\item\label{bib:Altschuler}
Altschuler E. L. et al. Possible global minimum lattice configurations for Thomson's problem of charges on a sphere //Physical Review Letters. – 1997. – Т. 78. – №. 14. – С. 2681.

\item\label{bib:Barber}
Barber C. B., Dobkin D. P., Huhdanpaa H. Qhull: Quickhull algorithm for computing the convex hull //Astrophysics Source Code Library. – 2013.

\item\label{bib:Wales}
Wales D. J. et al. The Cambridge cluster database. – 2001.

\item\label{bib:Bondarenko}
Bondarenko A. N., Karchevskiy M. N., Kozinkin L. A. The Structure of Metastable States in The Thomson Problem //Journal of Physics: Conference Series. – IOP Publishing, 2015. – Т. 643. – №. 1. – С. 012103.

\end{enumerate}

