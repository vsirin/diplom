\documentclass[a4paper,12pt]{article}
\usepackage{mathtext}
\usepackage{cmap}					% поиск в PDF
\usepackage[T2A]{fontenc}			% кодировка
\usepackage[utf8x]{inputenc}			% кодировка исходного текста
\usepackage[english,russian]{babel}	% локализация и переносы
\usepackage{indentfirst}
\usepackage{amsfonts}
\usepackage{amsthm}
\usepackage{amsmath}
\usepackage{graphicx}
\usepackage{indentfirst}
\usepackage{amscd,amssymb}
\usepackage{amsfonts,amsmath,array}
\usepackage{mathtools}
\DeclarePairedDelimiter\ceil{\lceil}{\rceil}
\DeclarePairedDelimiter\floor{\lfloor}{\rfloor}
\newtheorem{definition}{Определение}
\newtheorem{theorem}{Теорема}
\newtheorem{lemma}{Лемма}
\usepackage{titlesec}
\titleformat{\section}[block]{\Large\bfseries\centering}{\thesection}{1ex}{}

\usepackage{geometry} % Меняем поля страницы
\geometry{left=3cm}% левое поле
\geometry{right=1cm}% правое поле
\geometry{top=2cm}% верхнее поле
\geometry{bottom=2cm}% нижнее поле
\pagestyle{empty}

\usepackage{setspace}
\usepackage[14pt]{extsizes}

\begin{document} 

\onehalfspacing

\maketitle

\newpage
\pagestyle{plain}
\section{ВВЕДЕНИЕ}
Граф в математике представляет из себя множество вершин и множество ребер, то есть соединений между парами вершин. Многие структуры, представляющие практический интерес в математике и информатике, могут быть представлены графами. Плоским дистанционным графом называется граф, вершины которого --- точки плоскости, а ребрами соединены вершины, находящиеся на евклидовом расстоянии~1. Хроматическое число графа --- это минимальное число цветов, в которое можно покрасить вершины этого графа так, чтобы вершины, соединенные ребрами, были покрашены в разные цвета (обозначим его за $\chi(G)$).
\subsection{О хроматическом числе плоскости}
 Хроматическое число плоскости --- минимальное число цветов, в которое можно покрасить плоскость так, чтобы точки на расстоянии~1 были покрашены в разные цвета (обозначим его за $\chi(\mathbb{R}^2)$). Так как вершины плоского дистанционного графа принадлежат плоскости, то хроматическим числом графа можно оценить снизу хроматическое число плоскости \cite{Rai}. Хорошо известная следующая теорема:
 \begin{theorem} \label{1}
 $4\leq \chi(\mathbb{R}^2)\leq7$
 \end{theorem}
 
  \begin{figure}[h]
    \center{\includegraphics[width=0.5\linewidth]{hex_plane1.png}}
    \caption{Раскраска плоскости в 7 цветов}
\end{figure}
 
 Получить эти оценки сравнительно несложно. Более чем за 60 лет этот результат не был улучшен. В 2018 г. Обри Ди Греем\cite{de Grey} были найдены первые примеры плоских дистанционных графов с хроматическим числом 5. В статье \cite{Heule} был найден граф с минимальным из известных на данный момент числом вершин $n = 553$. В работе \cite{Exoo} была предложена принципиально иная конструкция графа. Другие примеры, кроме перечисленных,  на текущий момент в литературе не описаны. Их нахождение может привести к дальнейшим продвижениям в задаче о хроматическом числе плоскости.
 \begin{theorem} \label{2}
 $\chi(\mathbb{R}^2)\geq 5$
 \end{theorem}
 
Доказательство основано на компьютерном переборе раскрасок графа на 1581 вершине.
Позднее был найден граф на 553 вершинах.
 
 \begin{figure}[h]
    \center{\includegraphics[width=0.8\linewidth]{pic_golomb.png}}
  \hfill
  \caption{Примеры дистанционных графов с хроматическим числом 4}
\end{figure}
\begin{figure}[h]
    \center{\includegraphics[width=0.6\linewidth]{graph_h.png}}
  \hfill
  \caption{Граф Обри Ди Грея}
\end{figure}
\subsection{Хроматические числа вещественных пространств}
В одномерном случае $\chi(\mathbb{R}) = 2$. При $n = 2$ получается в точности исходная формулировка вопроса о хроматическом числе плоскости. При $n = 3$ задача представляется еще более сложной. Последние оценки с обеих сторон получены в текущем столетии.

\begin{theorem} \label{3}
 $6\leq \chi(\mathbb{R}^3)\leq 15$
 \end{theorem}
 
 Нижняя оценка принадлежит О. Нечуштану \cite{{Nech}}, верхняя — Д. Кулсону \cite{Coul}.
 В асимптотике выполняются следующие оценки: 
 
 \begin{theorem} \label{4}
 $(1.239...+o(1))^n\leq \chi(\mathbb{R}^n)\leq (3+o(1))^n$
 \end{theorem}
 Нижняя оценка принадлежит А. М. Райгородскому \cite{Rai1}, верхняя — Д. Г. Ларману и К. А. Роджерсу \cite{Larm}. Асимптотические нижние оценки в этой и смежных задачах были получены линейно-алгебраическим методом, который интересен сам по себе.

\subsection{Хроматические числа рациональных пространств}
Одномерный случай тривиален: $\chi(\mathbb{Q}) = 2$. Как ни странно, точное значение хроматического числа $\mathbb{Q}^n$ известно не только в размерности $2$, но и в размерностях $3$ и $4$.

\begin{theorem} \label{5}
 $\chi(\mathbb{Q}^2) = \chi(\mathbb{Q}^3) = 2, \chi(\mathbb{Q}^4) = 4 $
 \end{theorem}
 
 Лучшая асимптотическая нижняя оценка принадлежит А. М. Райгородскому \cite{Rai1}, верхняя — Д. Г. Ларману и К. А. Роджерсу \cite{Larm}. 

\begin{theorem} \label{6}
 $(1.199...+o(1))^n\leq \chi(\mathbb{Q}^n)\leq (3+o(1))^n$
 \end{theorem}
 
 \subsection{Хроматическое число сферы}
 Аналогично можно определить хроматическое число двумерной сферы и сфер больших размерностей. Расстояние при этом считается евклидовым. При этом важным параметром становится радиус сферы.
Очевидно, что $\chi(S_r^n) = 1$ при $r < \frac{1}{2}$.
 Будет разумным определять хроматическое число сферы при $r \geq \frac{1}{2}$. В статье \cite{Simmons} получены следующие результаты:
 
 \begin{theorem} \label{7}
 $4 \leq \chi(S_r)$ \text{при} $\frac{1}{\sqrt{3}} \leq r.$ \text{И эта оценка точна, т.к.} $\chi(S_{\frac{1}{\sqrt{2}}}) = 4$
 \end{theorem}
 
 В 1983 г.  Л. Ловас доказал, что  $\chi(S_{r}^{n-1}) \geq n.$ В статье \cite{Rai_sphere} А.М. Райгородским показал, что хроматическое число сферы растет экспоненциально при росте размерности для всех $r > \frac{1}{2}$. 
  
 \begin{theorem} \label{8}
\text{Если} $r \in (\frac{1}{2}, \frac{1}{\sqrt{2}}]$, \text{то} $\chi(S_{r}^{n-1}) \geq \Bigg(2(\frac{1}{8r^2})^{\frac{1}{8r^2}}(1-\frac{1}{8r^2})^{1-\frac{1}{8r^2}}+o(1)\Bigg)^n$.

\text{Если} $r \geq \frac{1}{\sqrt{2}}$, \text{то} $\chi(S_{r}^{n-1}) \geq \Bigg(2(\frac{1}{4})^{\frac{1}{4}}(\frac{3}{4})^{\frac{3}{4}}+o(1)\Bigg)^n$.
 \end{theorem}
 
 Очевидно, что $\chi(S_r^{n-1}) \leq \chi(\mathbb{R}^n) \leq (3+o(1))^n$, т.к. $S_r^{n-1}$ лежит в $\mathbb{R}^n$. В общем случае лучших верхних оценок нет. Однако К.А. Роджерс \cite{Rogers} получил более точную оценку в случае $r < \frac{3}{2}$.
 
 \begin{theorem} \label{9}
\text{Если} $r  < \frac{3}{2},$ \text{то} $\chi(S_r^{n-1}) \leq (2r+o(1))^n$.
\end{theorem}

В работе \cite{Pros} Р. Просановым получена следующая верхняя оценка хроматического числа сферы:

 \begin{theorem} \label{10}
\[ \chi(S_r^{n-1}) \leq (x(r)+o(1))^n,\: \text{где} \: x(r) = \begin{cases}\sqrt{5-\frac{2}{r}+4\sqrt{1-\frac{5r^2-1}{4r^4}}}, \quad r > \frac{\sqrt{5}}{2} \\ 

2r, \quad \frac{1}{2} < r \leq  \frac{\sqrt{5}}{2}
\end{cases}\]
\end{theorem}

\subsection{Измеримое хроматическое число}
 Измеримое хроматическое число пространства $\mathbb{R}^n$ --- минимальное число $m$ такое, что $\mathbb{R}^n$ можно разбить на $m$ частей, измеримых по Лебегу, так, что точки $\mathbb{R}^n$, лежащие на расстоянии 1, попадут в разные части (обозначим его за $\chi_m(\mathbb{R}^n))$. Одна из причин изучать измеримое хроматическое число состоит в том, что становятся доступны более сильные аналитические инструменты.
 
 Изучение измеримого хроматического числа было начато Фальконером \cite{Falc}, который доказал, что $\chi_m(\mathbb{R}^2) > 5$. На тот момент это было больше, чем известная оценка снизу для $\chi(\mathbb{R}^2)$.
 
 Хроматическое число пространство связано с хроматическим числом единичной сферы $S^{n-1} = \{x \in \mathbb{R}^2 : x \cdot x = 1\}.$ При $ -1 < t < 1$ рассмотрим граф $G(n,t)$, чьи вершины --- это точки $S^{n-1}$, и две вершины соединены ребром, если их скаkярное произведение $x \cdot y$ равно $t$. Хроматическое число $G(n,t)$ и его измеримая версия обозначаются как $\chi(G(n,t))$ и $\chi_m(G(n,t))$ соответственно и определяются аналогично пространственному случаю.
Хроматическое число этого графа было изучено Л.Ловасом \cite{Lov}, в частности в том случае, когда $t$ мало.


\begin{theorem} \label{11}
$n \leq \chi(G(n,t))$ при $-1 < t < 1$,

$\chi(G(n,t)) \leq n + 1$ при $-1 < t \leq -1/n$.

\end{theorem}

П. Франкл и Р. Уилсон \cite{FrW} получили следующий результат для ортогональных векторов.

 \begin{theorem} \label{12}

$(1+o(1))(1.13)^n \leq \chi_m(G(n,0)) \leq 2^{n-1} $.

\end{theorem}

Хроматическое число $G(n,t)$ (и его измеримая версия) дает нижнюю оценку для хроматического числа $\mathbb{R}^n$. После соответствующего сжатия, каждая правильная покраска $\mathbb{R}^n$, пересеченная с $S^{n-1}$, даст правильную покраску графа $G(n,t)$, а операция пересечения сохраняет измеримость по Лебегу.

Неизвестно, верно ли, что $\chi_m(\mathbb{R}^n) = \chi(\mathbb{R}^n)$. В работе \cite{Bach} было показано, что $(1.268... +o(1))^n \leq \chi_m(\mathbb(R)^n) \leq (3 + o(1))^n $. Таким образом, асимптотические нижние оценки этих величин сильно разнятся.

\subsection{Хроматическое число плоскости Лобачевского}
Для плоскости Лобачевского ставится аналогичная задача отыскания хроматического числа, однако, в отличие от хроматического числа сферы, где важным параметром был радиус сферы, в этом случае параметром становится запрещенное расстояние $d$. Хроматическое число обозначается как $\chi(\mathbb{H},d).$

Изучение $\chi(\mathbb{H},d)$ при росте $d$ можно сравнить с изучением $\chi(\mathbb{R}^n)$ при росте $n$, которое, как известно, растет экспоненциально. Однако доказательство, аналогичное случаю пространства \cite{Klo}, дает нижнюю оценку $4$ для $\chi(\mathbb{H},d)$.

В работе \cite{PP} получены следующие верхние оценки.
\begin{theorem}
$\chi(\mathbb{H},d) \leq 9$ при $d \leq 2 \ln{2}$.
 
$\chi(\mathbb{H},d) \leq 12$ при $d \leq 2 \ln{3}$.
  
$\chi(\mathbb{H},d) \leq 5(\ceil{\frac{d}{\ln{4}}})$ при $d > 2 \ln{3}$.
\end{theorem}

В работе \cite{CG} получена более сильная нижняя оценка в предположении, что разбивать плоскость можно только на измеримые части. 
\begin{theorem}
Пусть дано $d > 0$. Предположим, что $C_1, \dots, C_k$ --- измеримые (по Хаару) подмножества $\mathbb{H}$ такие, что $\mathbb{H} =  \cup_i C_i$, и в каждом $C_i$ нет точек на расстоянии $d$. Тогда $k \geq 6$ при достаточно большом $d$.
\end{theorem}

При этом остается вопрос, получится ли тем же методом доказать, что $\chi_m(\mathbb{H},d) \rightarrow \infty$ при $d \rightarrow \infty$ или же как-то улучшить нижнюю оценку $4$ для обычного хроматического числа?
 
  \begin{thebibliography}{99}
 \bibitem{Nech}
Oren Nechushtan. On the space chromatic number. Discrete mathematics, 256(1):499–507, 2002.
\bibitem{Coul}
 David Coulson. A 15-colouring of 3-space omitting distance one. Discrete mathematics, 256(1):83–90, 2002.
 \bibitem{Rai1}
 Андрей М. Райгородский. О хроматическом числе пространства. Успехи математических наук, 55(2):147–148, 2000.
 \bibitem{Larm}
  David G. Larman and Ambrose C. Rogers. The realization of distances within sets in Euclidean space. Mathematika, 19(01):1–24, 1972. 
\bibitem{Rai}
Райгородский А. М. Хроматические числа //М.: МЦНМО. – 2003.
\bibitem{de Grey}
de Grey A. D. N. J. The chromatic number of the plane is at least 5 //arXiv preprint arXiv:1804.02385. – 2018.
\bibitem{Heule}
Heule M. J. H. Computing small unit-distance graphs with chromatic number 5 //arXiv preprint arXiv:1805.12181. – 2018.
\bibitem{Exoo}
Exoo G., Ismailescu D. The Chromatic Number of the Plane is At Least 5: A New Proof //Discrete and Computational Geometry. – 2018. – С. 1-11.
\bibitem{Simmons}
Simmons G. J. The chromatic number of the sphere //Journal of the Australian Mathematical Society. – 1976. – Т. 21. – №. 4. – С. 473-480.
\bibitem{Rai_sphere}
Raigorodskii A. M. On the chromatic numbers of spheres in ℝ n //Combinatorica. – 2012. – Т. 32. – №. 1. – С. 111-123.
\bibitem{Rogers}
Rogers C. A. Covering a sphere with spheres //Mathematika. – 1963. – Т. 10. – №. 2. – С. 157-164.
\bibitem{Pros}
Prosanov R. Chromatic numbers of spheres //Discrete Mathematics. – 2018. – Т. 341. – №. 11. – С. 3123-3133.
\bibitem{Falc}
Falconer K. J. The realization of distances in measurable subsets covering Rn //Journal of Combinatorial Theory, Series A. – 1981. – Т. 31. – №. 2. – С. 184-189.
\bibitem{Lov}
Lovász L. Self-dual polytopes and the chromatic number of distance graphs on the sphere //Acta Sci. Math.(Szeged). – 1983. – Т. 45. – №. 1-4. – С. 317-323.
\bibitem{FrW}
Frankl P., Wilson R. M. Intersection theorems with geometric consequences //Combinatorica. – 1981. – Т. 1. – №. 4. – С. 357-368.
\bibitem{Bach}
Bachoc C., Passuello A., Thiery A. The density of sets avoiding distance 1 in Euclidean space //Discrete & Computational Geometry. – 2015. – Т. 53. – №. 4. – С. 783-808.
\bibitem{Klo}
Kloeckner B. Coloring distance graphs: a few answers and many questions //arXiv preprint arXiv:1305.2765. – 2013.
\bibitem{PP}
Parlier H., Petit C. Chromatic numbers for the hyperbolic plane and discrete analogs //arXiv preprint arXiv:1701.08648. – 2017.
\bibitem{CG}
DeCorte E., Golubev K. Lower bounds for the measurable chromatic number of the hyperbolic plane //Discrete & Computational Geometry. – 2019. – Т. 62. – №. 2. – С. 481-496.
\end{thebibliography}


\end{document}

%Статья Симмонса:
%https://www.cambridge.org/core/services/aop-cambridge-core/content/view/BA3C1CF470D0226780ED33D06FD706CC/S1446788700019315a.pdf/chromatic_number_of_the_sphere.pdf
% https://arxiv.org/pdf/1201.0486.pdf

%сферические треугольники
%https://encrypted-tbn0.gstatic.com/images?q=tbn%3AANd9GcQ1QiS0b37xdK4ga-vXgIuRVwdy61bH_ZrIk4FifjMGTuOem-aP&usqp=CAU

% асимптотика при растущем n
% Райгородский, 2012
%https://link.springer.com/content/pdf/10.1007/s00493-012-2709-9.pdf

% Костина, Райгородский, 2015
% https://link.springer.com/content/pdf/10.1134/S1064562415040298.pdf
% https://link.springer.com/content/pdf/10.1134/S0001434619010036.pdf

% Просанов, 2017
%https://arxiv.org/pdf/1711.03193.pdf


% измеримое хроматическое число 
% https://link.springer.com/content/pdf/10.1007/s00039-009-0013-7.pdf
% https://mipt.ru/upload/medialibrary/b26/5-10.pdf


%хроматические числа плоскости Лобачевского

% https://arxiv.org/pdf/1701.08648.pdf

% измеримое хроматическое число плоскости Лобачевского
% https://link.springer.com/article/10.1007/s00454-018-0027-8


% Канель, Воронов, Черкашин
% https://arxiv.org/pdf/1512.06444.pdf